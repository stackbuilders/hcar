\documentclass[DIV16,twocolumn,10pt]{scrreprt}
\usepackage{paralist}
\usepackage{graphicx}
\usepackage[final]{hcar}

%include polycode.fmt

\begin{document}

\begin{hcarentry}{openssh-github-keys (Stack Builders)}
\report{Stack Builders}
\status{A library to automatically manage SSH access to servers using GitHub teams}
\participants{Justin Leitgeb}% optional
\makeheader

It is common to control access to a Linux server by changing public
keys listed in the \texttt{authorized\_keys} file. Instead of
modifying this file to grant and revoke access, a relatively new
feature of OpenSSH allows the accepted public keys to be pulled from
standard output of a command.

This package acts as a bridge between the OpenSSH daemon and GitHub so
that you can manage access to servers by simply changing a GitHub
Team, instead of manually modifying the \texttt{authorized\_keys}
file. This package uses the
\href{http://hackage.haskell.org/package/octohat}{Octohat} wrapper
library for the GitHub API which we recently released.

openssh-github-keys is still experimental, but we are using it on a
couple of internal servers for testing purposes. It is available on
\href{http://hackage.haskell.org/package/openssh-github-keys}{Hackage}
and contributions and bug reports are welcome in the
\href{https://github.com/stackbuilders/openssh-github-keys}{GitHub
  repository}.

While we don't have immediate plans to put openssh-github-keys into
heavier production use, we are interested in seeing if community
members and system administrators find it useful for managing server
access.

\FurtherReading
  \url{https://github.com/stackbuilders/openssh-github-keys}
\end{hcarentry}

\end{document}
