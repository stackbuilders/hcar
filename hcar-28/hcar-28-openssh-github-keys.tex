\documentclass[DIV16,twocolumn,10pt]{scrreprt}
\usepackage{paralist}
\usepackage{graphicx}
\usepackage[final]{hcar}

%include polycode.fmt

\begin{document}

\begin{hcarentry}{openssh-github-keys (Stack Builders)}
\report{Stack Builders}
\status{Open Source, active library to automatically manage SSH access to servers using your existing GitHub teams}
\participants{Justin Leitgeb}% optional
\makeheader

A common DevOps task is adding a new member’s SSH key to a remote server. The
usual workflow is as follows:

\begin{itemize}
  \item Ask the new member for their public SSH key
  \item Copy and paste the new member SSH key inside the authorized_keys file
  \item Finally the new member should be able to SSH to the server
\end{itemize}

Although this process doesn't take too long and doesn't seem too complicated,
managing dozens of servers with several users it quickly becomes cumbersome -
that's why we decide to write a library in Haskell that fetch the public SSH
keys from GitHub so we can use them to authorize access to the remote servers.

Previous versions of OpenSSH have an option that allows you to pull a list of
authorized keys for a user. This command pulls keys GitHub and OpenSSH allows
login using the selected user accounts.

In latest version we introduce the ability to read all the configuration from a
file since openssh configuration value AuthorizedKeysCommand only takes a
single command without parameters.

Since openssh-github-keys is just an experimental library, you may want to have
a user account that relies on a standard authorized_keys file for a small group
of primary users (e.g., system administrators) and give the rest of your team
access through the GitHub authentication mechanism.

\FurtherReading
  \url{https://github.com/stackbuilders/openssh-github-keys}
\end{hcarentry}

\end{document}
